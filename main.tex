\documentclass{article}
\usepackage[german]{babel}
\usepackage[utf8]{inputenc}
\usepackage{url}
\usepackage{multicol}
\usepackage{amsmath}
\usepackage{esint}
\usepackage{amsfonts}
\usepackage{tikz}
\usetikzlibrary{decorations.pathmorphing}
\usepackage{amsmath,amssymb}
\usepackage{fancyhdr}
\usepackage{tabularx}

\usepackage{graphicx}
\usepackage{wrapfig}

\usepackage{titlesec}
\titleformat{\section}
  {\normalfont\fontsize{10}{15}\bfseries}{\thesection}{1em}{}


\usepackage{colortbl}
\usepackage{xcolor}
\usepackage{mathtools}
\usepackage{amsmath,amssymb}
\usepackage{enumitem}
\makeatletter

\newcommand*\bigcdot{\mathpalette\bigcdot@{.5}}
\newcommand*\bigcdot@[2]{\mathbin{\vcenter{\hbox{\scalebox{#2}{$\m@th#1\bullet$}}}}}
\makeatother

\usepackage{geometry}
 \geometry{
 landscape,
 a4paper,
 left=5mm,
 right=5mm,
 top=5mm,
 bottom=15mm,
 }



\pagestyle{fancy}
\fancyhead{} % clear all header fields
\renewcommand{\headrulewidth}{0pt} % no line in header area
\fancyfoot{} % clear all footer fields
\lfoot{Martin Oswald}
\cfoot{Zusammenfassung STS}
\rfoot{\thepage}
\parindent0pt
\parskip2pt


% --------------- define mybox -----------------
\tikzstyle{mybox} = [draw=black, fill=white, very thick,
    rectangle, rounded corners, inner sep=10, inner ysep=10]
\tikzstyle{fancytitle} =[fill=black, text=white, font=\bfseries]


\begin{document}
% --------------------------------------------------------------------
% deskriptive Statistik
% --------------------------------------------------------------------
\section*{Deskriptive Statistik}

\begin{multicols}{3}

% grundbegriffe
\begin{tikzpicture}
\node [mybox] (box){
    \begin{minipage}{0.295\textwidth}
    \begin{tabular}{lp{6.5cm} l}
    	PMF: & $f(x)$ Relative Häufigkeit (Stabdiagramm)\\
    	CMF: & $F(x)$ Kumulative relative Häufigkeit (Treppendiagramm)\\
    	PDF: & $f(x)$ Höhe Balken Histogramm\\
    	CDF: & $F(x)$ Kulutaive Fläche Balken Histogramm\\
    	$h_i$: & Absolute häufigkeit\\
    	$f_i$: & Relative häufigkeit\\
    	$x_{med}$: & 2. Quantil oder $R_{0.5} $\\
    	$x_{mod}$: & $Modus$ oder $Modalwert$ ist der häufigste Stichprobenwert\\
    	$\overline{x}$: & arithmetisches Mittel\\
    	$s_x^2$: & Varianz\\
    	$s_x$: & Standardabweichung\\
    	$s_{kor}$: & korrigierte Standardabweichung
	\end{tabular}
    
    \end{minipage}
};
\node[fancytitle, right=10pt] at (box.north west) {Grundbegriffe};
\end{tikzpicture}

% Quantile
\begin{tikzpicture}
\node [mybox] (box){
    \begin{minipage}{0.295\textwidth}
    Eine Zahl $R$ ist genau dann ein q-Quantil, falls sie die Stichprobe in zwei Teile aufteilt.\\
    \\
    \begin{tabular}{lp{6.5cm} l}
    	$R_{0.25}$: & 1. Quantil\\
    	$R_{0.50}$: & 2. Quantil oder Median\\
    	$R_{0.75}$: & 3. Quantil\\
	\end{tabular}
    \begin{tabular}{lp{6.5cm} l}
    $n \cdot q$ ist eine \textbf{ganze} zahl:
    & $R_q = \frac{1}{2}(x_nq + x_{nq + 1})$\\
    $n \cdot q$ ist \textbf{keine ganze} zahl:
    & $R_q = x_{\lceil np \rceil}$\\
	\end{tabular}
	\\
    \end{minipage}
};
\node[fancytitle, right=10pt] at (box.north west) {Quantile};
\end{tikzpicture}

% Quantile aus Funktion
\begin{tikzpicture}
\node [mybox] (box){
    \begin{minipage}{0.295\textwidth}
	Quantile CMF:\\
	\includegraphics[width=\textwidth]{images/quantile_cmf.jpg}\\
	Quantile CDF:\\
	\includegraphics[width=\textwidth]{images/quantile_cdf.png}\\
	Zur Berechnung von $R_q$ sucht man diejenige Klasse $[a,b[$ mit $F(a) \leq q \leq F(b)$
	b
	\begin{itemize}
    \setlength\itemsep{0em}
        \item $m = \frac{q - F(a)}{R_q - a}$
        \item $R_q = a + (b - a)\cdot\frac{q - F(a)}{F(b) - F(a)}$
        \item $q = F(a) + \frac{F(b) - F(a)}{b - a}\cdot(R_q - a)$
        \item $q = F(a) +  f(R_q)\cdot(R_q - a)$
    \end{itemize}
    \end{minipage}
};
\node[fancytitle, right=10pt] at (box.north west) {Quantile aus Funktion};
\end{tikzpicture}


% boxplot
\begin{tikzpicture}
\node [mybox] (box){
    \begin{minipage}{0.295\textwidth}
    \includegraphics[width=\textwidth]{images/boxplot.png}\\
    Untere und obere Antenne sind Stichprobenwerte!
    \end{minipage}
};
\node[fancytitle, right=10pt] at (box.north west) {Boxplot};
\end{tikzpicture}


\begin{tikzpicture}
\node [mybox] (box){
    \begin{minipage}{0.295\textwidth}
    \begin{tabular}{lp{6.5cm} l}
        $n$: & Anzahl Stichprobenelemente\\
        $m$: & Anzahl Merkmahle\\
        $a_i$: & Stichprobenwert\\
	\end{tabular}
    \begin{itemize}
    \setlength\itemsep{0em}
        \item $\overline{x} = \frac{1}{n}\sum\limits_{i=1}^{n} x_i = \frac{1}{n}\sum\limits_{i=0}^{m} a_i 
        \cdot h_i = \sum\limits_{i=1}^{m} a_i \cdot f_i$
        \item $s_x^2 = \frac{1}{n}\sum\limits_{i=1}^{n} {(x_i - \overline{x})}^2 =
        [\sum\limits_{i=1}^{m} a_i^2 \cdot f_i] - \overline{x}^2$
        \item $s_x = \sqrt{s_x^2}$
        \item $s_{kor}^2 = \frac{1}{n-1}\sum\limits_{i=1}^{n} {(x_i - \overline{x})}^2 =
        \frac{n}{n-1}\cdot s_x^2$
        \item $s_{kor} = \sqrt{s_{kor}^2} = \sqrt{\frac{n}{n-1}}\cdot s_x$
    \end{itemize}
    \end{minipage}
};
\node[fancytitle, right=10pt] at (box.north west) {Lage- und Streumasse};
\end{tikzpicture}


\begin{tikzpicture}
\node [mybox] (box){
    \begin{minipage}{0.295\textwidth}
    \includegraphics[width=\textwidth]{images/bsp_histogramm.png}
    \end{minipage}
};
\node[fancytitle, right=10pt] at (box.north west) {Bsp. klassierte Stichprobe};
\end{tikzpicture}
\end{multicols}
\newpage








% --------------------------------------------------------------------
% deskriptive Statistik mit mehrere Merkmale
% --------------------------------------------------------------------
\section*{Deskriptive Statistik mit mehrere Merkmale}
\begin{multicols}{3}

% grundbegriffe
\begin{tikzpicture}
\node [mybox] (box){
    \begin{minipage}{0.295\textwidth}
    \begin{tabular}{lp{6.5cm} l}
        $s_{xy}$: & Kovarianz\\
        $r_{xy}$: & Korrelationskoeffizient oder normierte Kovarianz (Pearson)\\
        $r_{sp}$: & Korrelationskoeffizient nach Spearman oder Rangkorrelationskoeffizient\\
        $s_x$: & Standardabweichung der $x_i$ Werte\\
        $s_y$: & Standardabweichung der $y_i$ Werte\\
        $\overline{x}$: & Arithmetisches Mittel der $x_i$\\
        $\overline{y}$: & Arithmetisches Mittel der $y_i$
	\end{tabular}
    \end{minipage}
};
\node[fancytitle, right=10pt] at (box.north west) {Grundbegriffe};
\end{tikzpicture}


% Mittelwerte
\begin{tikzpicture}
\node [mybox] (box){
    \begin{minipage}{0.295\textwidth}
    \begin{multicols}{2}
    \begin{itemize}
    \setlength\itemsep{0em}
        \item $\overline{x} =\frac{1}{n}\sum\limits_{i=1}^{n} x_i$
        \item $\overline{y} =\frac{1}{n}\sum\limits_{i=1}^{n} y_i$
    \end{itemize}
    \begin{itemize}
    \setlength\itemsep{0em}
        \item $\overline{xy} =\frac{1}{n}\sum\limits_{i=1}^{n} x_i y_i$
        \item $\overline{x^2} =\frac{1}{n}\sum\limits_{i=1}^{n} x_i^2$
        \item $\overline{y^2} =\frac{1}{n}\sum\limits_{i=1}^{n} y_i^2$
    \end{itemize}
    \end{multicols}
    \end{minipage}
};
\node[fancytitle, right=10pt] at (box.north west) {Mittelwerte};
\end{tikzpicture}


% Korrelation Pearson
\begin{tikzpicture}
\node [mybox] (box){
    \begin{minipage}{0.295\textwidth}
    \begin{itemize}
    \setlength\itemsep{0em}
        \item $s_x^2 = \overline{x^2} - \overline{x}^2$
        \item $s_x = \sqrt{s_x^2}$
        \item $s_x^2 = \overline{x^2} - \overline{x}^2$
        \item $s_x = \sqrt{s_x^2}$
        \item $s_{xy} = \frac{1}{n}\sum\limits_{i=1}^{n} (x_i - \overline{x})(y_i - \overline{y})$
        \item $s_{xykor} = \frac{1}{n-1}\sum\limits_{i=1}^{n} (x_i - \overline{x})(y_i - \overline{y})$
        \item $r_{xy} = \frac{s_{xy}}{s_x \cdot s_y} = \frac{s_{xykor}}{s_{xkor} \cdot s_{ykor}}$
    \end{itemize}
    $r_{xy}$ ist im Bereich $[0,1]$. desto näher an $1$, desto grösser ist die Korrelation
    \end{minipage}
};
\node[fancytitle, right=10pt] at (box.north west) {Korrelation (Pearson)};
\end{tikzpicture}


% Korrelation Bsp
\begin{tikzpicture}
\node [mybox] (box){
    \begin{minipage}{0.295\textwidth}
    \includegraphics[width=\textwidth]{images/bsp_korrelation.png}
    \end{minipage}
};
\node[fancytitle, right=10pt] at (box.north west) {Bsp. Korrelation};
\end{tikzpicture}


% Korrelation Spearman
\begin{tikzpicture}
\node [mybox] (box){
    \begin{minipage}{0.295\textwidth}
    Bei dem Rankkorrelationskoeffizient werden die Ränge der Stichprobenwerte verwendet.
    um den Rang zu bestimmen können die Werte der Grösse nach sortiert werden.
    \begin{itemize}
    \setlength\itemsep{0em}
        \item $r_{sp} = \sum\limits_{i=1}^{n} {(rg(x_i)-\overline{rg(x_i})}^2$
    \end{itemize}
    Falls der gleiche Stichprobenwert mehrmals vorkommt muss der Rang addiert und durch die Anzahl Vorkomtnisse subtrahiert werden.\\
    Bsp. Verbundene Ränge:\\
    \includegraphics[width=\textwidth]{images/verbundene_raenge.png}
    \end{minipage}
};
\node[fancytitle, right=10pt] at (box.north west) {Korrelation (Spearman)};
\end{tikzpicture}
\end{multicols}
\newpage





% --------------------------------------------------------------------
% Kombinatorik
% --------------------------------------------------------------------
\section*{Kombinatorik}
\begin{multicols}{3}


% Problem
\begin{tikzpicture}
\node [mybox] (box){
    \begin{minipage}{0.295\textwidth}
    \begin{itemize}
    \setlength\itemsep{0em}
        \item \textbf{Zahlenschlossproblem}: Sie haben den Zahlencode ihres Zahlenschlosses vergessen. Das Schloss besteht aus 6 Zahlenkränzen mit den Zahlen 0 bis 9. Wie viele Einstellungen müssen Sie im schlimmsten Fall probieren, um ihren Zahlencode zu finden?
        \item \textbf{Schwimmwettkampf}: Bei einem Schwimmwettbewerb starten 10 Schwimmerinnen. Wie viele mögliche Platzierungen gibt es, wenn Sie nur die ersten drei Plätze betrachten und nicht zulassen, dass Schwimmerinnen zeitgleich im Ziel ankommen können?
        \item \textbf{Lotto}: Wie gross sind die Chancen beim Lotto 6 aus 49 mit einem Versuch sechs richtige Zahlen vorauszusagen?
        \item \textbf{Bitproblem}: Wie viele verschiedene natürliche Zahlen können mit 64 Bits binär dargestellt werden?
        \item \textbf{Zahnarztproblem}: Eine Zahnärztin erlaubt den Kindern, nach der Behandlung zur Belohnung 3 Spielzeuge aus 5 Töpfen auszusuchen. Die 5 Töpfe sind dabei jeweis mit einer Art Spielzeug befüllt (Gummiball, Spielfigur, Toyauto, Jojo, Kreisel). Wie viele verschiedene Möglichkeiten hat ein Kind?
        \item \textbf{Fussballmannschaft}: Aus einer Klasse mit 20 Studierenden soll eine Fussballmannschaft mit 11 Spielern zusammengestellt werden. (6a) Wie viele Möglichkeiten gibt es? (6b) Wie viele Möglichkeiten gibt es, wenn die Mannschaft genau aus 6 Frauen und 5 Männern bestehen soll und die Klasse aus 8 Frauen und 12 Männern besteht.
    \end{itemize}
    \end{minipage}
};
\node[fancytitle, right=10pt] at (box.north west) {Probleme};
\end{tikzpicture}


% Problem
\begin{tikzpicture}
\node [mybox] (box){
    \begin{minipage}{0.295\textwidth}
    \begin{itemize}
    \setlength\itemsep{0em}
        \item \textbf{Buchstabenproblem}: Mit 10 verschiedenen Buchstaben sollen Wörter von 5 Zeichen gebildet werden. (7a) Wie viele solche Wörter gibt es, wenn dabei kein Buchstabe doppelt vorkommen darf. (7b) Wie viele verschiedene Wörter gibt es, wenn Buchstaben mehrmals vorkommen können.
        \item \textbf{Tellschiessen}: Wilhelm Tell schiesst mit drei Pfeilen auf eine Zielscheibe, welche in 10 ringförmige Bereiche unterteilt ist. Wie viele verschiedene Resultate gibt es für Wilhelm?
        \item \textbf{Napoleon}: schart seine 10 Generäle für eine Beratung um sich an einem kreisrunden Tisch mit 11 Plätzen. (9a) Wie viele verschiedene Sitzreihenfolgen gibt es? (9b) Wie viele verschiedene Sitzreihenfolgen gibt es, wenn Napoleons Liebling Marschall Ney immer an Napoleons Seite sitzen soll?
        \item \textbf{Gruppen}: Wie viele verschiedene Personengruppen kann man aus einer Klasse mit 20 Studierenden bilden?
        \item \textbf{Teilmengen}: (11a) Wie viele dreielementige Teilmengen hat die Menge {1,2,3,4}? (11b) Wie viele Teilmengen hat die Menge {1,2,3,4}? Wie viele Teilmengen hat eine Menge mit n Elementen?
    \end{itemize}
    \end{minipage}
};
\node[fancytitle, right=10pt] at (box.north west) {Probleme};
\end{tikzpicture}


% Binominalkoeffizient
\begin{tikzpicture}
\node [mybox] (box){
    \begin{minipage}{0.295\textwidth}
    \textbf{Definition:}
        \begin{itemize}
    \setlength\itemsep{0em}
        \item $\left(\begin{array}{c} n \\ k \end{array}\right) = \frac{n!}{(n-k)! \cdot k!}$
    \end{itemize}
    \textbf{Eigenschaften:}\\
        \begin{tabular}{lp{6.5cm} l}
            Leere Menge: & $\left(\begin{array}{c} n \\ 0 \end{array}\right) = 1$\\\\
            Symmetrie: & $\left(\begin{array}{c} n \\ k \end{array}\right) 
            = \left(\begin{array}{c} n \\ n-k \end{array}\right)$\\\\
            Rekursion: & $\left(\begin{array}{c} n+1 \\ k+1 \end{array}\right)
            = \left(\begin{array}{c} n \\ k \end{array}\right) +
            \left(\begin{array}{c} n \\ k+1 \end{array}\right)$\\\\
            Summe: & $\sum\limits_{k=0}^{n} \left(\begin{array}{c} n \\ k \end{array}\right)
            = 2^n$
	    \end{tabular}
    \end{minipage}
};
\node[fancytitle, right=10pt] at (box.north west) {Binominalkoeffizient};
\end{tikzpicture}


% Abzählprobleme
\begin{tikzpicture}
\node [mybox] (box){
    \begin{minipage}{0.295\textwidth}
    \includegraphics[width=\textwidth]{images/Diagramm_kombinatorik.png}
    \end{minipage}
};
\node[fancytitle, right=10pt] at (box.north west) {Abzählprobleme};
\end{tikzpicture}
\end{multicols}
\newpage





% --------------------------------------------------------------------
% Kombinatorik
% --------------------------------------------------------------------
\section*{Elementare Wahrscheinlichkeitsrechnung}
\begin{multicols}{3}

% grundbegriffe
\begin{tikzpicture}
\node [mybox] (box){
    \begin{minipage}{0.295\textwidth}
    \begin{tabular}{lp{6.5cm} l}
        $\Omega$: & Ergebnisraum. Alle möglichen ergebnisse eines Zufallsexperiments.\\
        $\rho$: & Zähldichte (PMF). $\rho$ ist eine Funktion die angibt mit welcher Wahrscheinlichkeit die möglichen Ergebnisse des Zufallsexperiments auftreten.
	\end{tabular}
    \end{minipage}
};
\node[fancytitle, right=10pt] at (box.north west) {Grundbegriffe};
\end{tikzpicture}


\begin{tikzpicture}
\node [mybox] (box){
    \begin{minipage}{0.295\textwidth}
        \begin{itemize}
        \setlength\itemsep{0em}
            \item (Unmögliches ergebnis) $P({})=0$
            \item (Sicheres ergebnis) $P(\Omega)=1$
            \item (Komplementäres Ereignis) $P(\Omega \setminus A)= 1 - P(A)$
            \item (Vereinigung) $P(A \cup B) = P(A) + P(B) - P(A \cap B)$
            \item (Sigma-Additivität) \\$P(A_1 \cup A_2 \cup A_3 \cup \ldots) = P(A_1)
            + P(A_2) + P(A_3) + \ldots$
            \item Ist $(\Omega , P)$ ein Laplace-Raum, so gilt: $P(M)= \frac{|M|}{|\Omega|}$
        \end{itemize}
    \end{minipage}
};
\node[fancytitle, right=10pt] at (box.north west) {Diskreter Wahrscheinlichkeitsraum};
\end{tikzpicture}


\begin{tikzpicture}
\node [mybox] (box){
    \begin{minipage}{0.295\textwidth}
        \begin{itemize}
        \setlength\itemsep{0em}
            \item \textbf{Linearität des Erwartungswertes:}\\
            $E(X+Y) = E(X)+E(Y)$ und $E(\alpha X) = \alpha E(X)$
            \item \textbf{Verschiebungssatz für die Varianz:}\\
            $V(X) = E(X^2) - {E(X)^2}$
        \end{itemize}
    \end{minipage}
};
\node[fancytitle, right=10pt] at (box.north west) {Kenngrössen};
\end{tikzpicture}


\begin{tikzpicture}
\node [mybox] (box){
    \begin{minipage}{0.295\textwidth}
        Die Wahrscheinlichkeit für das Eintreten des Ereignisses B, 
        unter Voraussetzung dass das Ereignis A eintritt
        \begin{itemize}
        \setlength\itemsep{0em}
            \item \textbf{Multiplikationssatz} \\
            $P(A \cap B) = P(A) \cdot P(B|A) = P(B) \cdot P(A|B)$
            \item \textbf{Satz von der totalen Wahrscheinlichkeit}\\
            $P(B) = P(A) \cdot P(B|A) + P(\overline{A}) \cdot P(B|\overline{A})$
            \item \textbf{Satz von Bayes}\\
            $P(B|A) = \frac{P(A \cap B)}{P(A)}$
        \end{itemize}
    \end{minipage}
};
\node[fancytitle, right=10pt] at (box.north west) {Bedingte Wahrscheinlichkeit};
\end{tikzpicture}

\begin{tikzpicture}
\node [mybox] (box){
    \begin{minipage}{0.295\textwidth}
        \includegraphics[width=\textwidth]{images/Bsp_Ereignisbaum_1.png}
        \includegraphics[width=\textwidth]{images/Bsp_Ereignisbaum_2.png}
    \end{minipage}
};
\node[fancytitle, right=10pt] at (box.north west) {Bsp. Ereignisbaum};
\end{tikzpicture}



\begin{tikzpicture}
\node [mybox] (box){
    \begin{minipage}{0.295\textwidth}
        Falls zwei Ereignisse stochastisch unabhängig sind, beeinflusst das
        Eintreten des einen Ereignisses nicht das Eintreten des Anderen. somit gilt:
        \begin{itemize}
        \setlength\itemsep{0em}
            \item $P(A|B) = P(A)$ und $P(B|A) = P(B)$
        \end{itemize}
        Für zwei Ereignisses $A$ und $B$ sind folgende Aussagen equivalent:
        \begin{itemize}
        \setlength\itemsep{0em}
            \item $A$ und $B$ sind stochastisch unabhängig
            \item $A$ und $\Omega \setminus B$ sind stochastisch unabhängig
            \item $\Omega \setminus A$ und $\Omega \setminus B$ sind stochastisch unabhängig
        \end{itemize}
        Für stochastisch unabhängige Zufallsvariabeln $X$,$Y$ gilt:
        \begin{itemize}
        \setlength\itemsep{0em}
            \item $E(X \cdot Y) = E(X) \cdot E(Y)$
            \item $V(X + Y) = V(X) + V(Y)$
        \end{itemize}
    \end{minipage}
};
\node[fancytitle, right=10pt] at (box.north west) {Stochastische Unabhängigkeit};
\end{tikzpicture}


\begin{tikzpicture}
\node [mybox] (box){
    \begin{minipage}{0.295\textwidth}
        \includegraphics[width=\textwidth]{images/bsp_stoch_abhängigkeit.png}
    \end{minipage}
};
\node[fancytitle, right=10pt] at (box.north west) {Bsp. Stochastische Unabhängigkeit};
\end{tikzpicture}


\end{multicols}
\newpage





% --------------------------------------------------------------------
% Spezielle Verteilungen
% --------------------------------------------------------------------
\section*{Spezielle Verteilungen}

\begin{multicols}{3}


% grundbegriffe
\begin{tikzpicture}
\node [mybox] (box){
    \begin{minipage}{0.295\textwidth}
    \begin{tabular}{lp{6.5cm} l}
        $\Omega$: & Ergebnisraum. Alle möglichen ergebnisse eines Zufallsexperiments.\\
        $\rho$: & Zähldichte (PMF). $\rho$ ist eine Funktion die angibt mit welcher Wahrscheinlichkeit die möglichen Ergebnisse des Zufallsexperiments auftreten.
	\end{tabular}
    \end{minipage}
};
\node[fancytitle, right=10pt] at (box.north west) {Grundbegriffe};
\end{tikzpicture}


\begin{tikzpicture}
\node [mybox] (box){
    \begin{minipage}{0.295\textwidth}
        \includegraphics[width=\textwidth]{images/diskrete_stetige_zufallsvariabeln.png}
    \end{minipage}
};
\node[fancytitle, right=10pt] at (box.north west) {Diskrete und stetige Zufallsvariabeln};
\end{tikzpicture}


\begin{tikzpicture}
\node [mybox] (box){
    \begin{minipage}{0.295\textwidth}
        Eine diskrete Zufallsvariable $X$ heisst hypergeometrisch verteilt mit Parametern: $n$ Objekte aus einer Gesamtheit von $N$ Objekten mit $M$ Merkmalsträgern und $N-M$ andersartigen Objekten, wenn Sie die folgende Verteilung besitzt:\\
        $P(X = x) = \frac{\left( \begin{array}{c} M \\ x \end{array} \right)
        \cdot \left( \begin{array}{c} N - M \\ n - x \end{array} \right)}{
            \left( \begin{array}{c} N \\ n \end{array} \right)}$
    \end{minipage}
};
\node[fancytitle, right=10pt] at (box.north west) {Hypergeometrische Verteilungen};
\end{tikzpicture}
    

\end{multicols}

\end{document}
